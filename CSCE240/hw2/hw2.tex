\documentclass[fleqn]{article}
\usepackage[left=1in, right=1in, top=1in, bottom=1in]{geometry}
\usepackage{mathexam}
\usepackage{verbatim}

\ExamClass{CSCE 240}
\ExamName{Homework 2}
\ExamHead{Due: 18 February 2020}

\let
\ds
\displaystyle

\begin{document}
\ExamInstrBox {
  You shall submit a zipped, \textbf{and only zipped}, archive of your homework
  directory, hw2. The directory shall contain, at a minimum, the files
  \texttt{parse\_scores.cc} and \texttt{parse\_scores.h}. Name the archive
  submission file hw2.zip \\

  I will use my own makefile to compile and link to your
  \texttt{parse\_scores.cc} and \texttt{parse\_scores.h} files. You must submit,
  at least, these two files.
}
%
This assignment tests your ability to parse data based on a provided format.
The premise is a text-based grade book with which you will interact using basic
file I/O. You will provide a small library of three functions:
\begin{itemize}
  \item \texttt{GetGrade}
  \item \texttt{GetMaxGradeStudentId}
  \item \texttt{GetAvgGrade}
\end{itemize}
%
The functions shall accept a character string which can be used to open a file
with an \texttt{fstream} object.
\\
%
\\
The file format is described as: number\_of\_students student\_id
number\_of\_grades grade\_0 grade\_1 grade\_n \\
\\
Read the provided header file documentation for instructions on how the
functions should work. \\
%
\\
I have provided you a basic test app which you can use to ensure that your code
is somewhat passing. I would suggest a much more rigorous testing scheme. Your
code must behave as indicated in the documentation. Passing the tests is not
enough to get full credit. You may call the test by using \\
\texttt{./test\_parse\_scores gradebook.txt} \\
%
\\
Late assignments will lose 10\% per day late, with no assignment begin accepted
after 3 days.\\
\\
%
Check your syllabus for the breakdown of grading.
\end{document}

