\documentclass[fleqn]{article}
\usepackage[left=1in, right=1in, top=1in, bottom=1in]{geometry}
\usepackage{mathexam}

\ExamClass{CSCE 240}
\ExamName{Homework 1}
\ExamHead{Due: 04 Feb 2020}

\let
\ds
\displaystyle

\begin{document}
\ExamInstrBox {
  You shall submit a zipped, \textbf{and only zipped}, archive of your homework
  directory, hw1. The directory shall contain, at a minimum, the file
  \texttt{binary\_converter.cc}. Your submission file must be named hw1.zip. \\

  I will use my own makefile to make your \texttt{binary\_converter.cc} file. Do
  not use a header for this assignment. My grader will not look for one.
}
%
Everyone in computing should have a basic understanding of binary integers. To
that end, you will write an app to convert decimal to binary.
%
\vspace{1.0em} \\
The application should perform as follows:
\begin{itemize}
	\item The program is called and runs with no prompt for input.

	\item The first value provided to the standard input stream shall be an
	integer and be interpreted as the number of bits of binary output. This
	value should be something in the range [1, 31]. Do be careful, sometimes I
	get careless and enter values outside this range. If that happens, the app 
	should quietly exit with a value of 1.

  \item The second value provided is the decimal integer to be converted.
  Though I can guarantee that you will see values in the range [$0,\
  2^{31}-1$], I am sometimes careless about the number of bits required to
  represent a decimal integer. If I make a mistake by entering a value too
  large for the number of bits, that is on me and you should still only print
  the number of bits requested. If I enter a negative value, do not print a
  two's complement; again, quietly exit, this time with a value of 2.

  \item An algorithm to convert decimal number, $n$,  to binary (without
    arrays):
    \begin{enumerate}
      \item Use loop to calculate the largest power of 2, call it $i$, in $n$,
        print a 1 and subtract $2^i$ from $n$.\footnote{Do not forget to check
        whether this exponent is less than or equal to the first integer passed
        to standard input. If it is outside that value, you should subtract as
        indicated, but not print. You should then find the next largest
        exponent and try again.}
      \item For each exponent, $j$, from $i - 1$ down to $0$,
        \begin{enumerate}
          \item if that $2^j \leq n$, print a $1$ and subtract $2^j$ from $n$
          \item otherwise if that $n < 2^j$, print a $0$.
        \end{enumerate}
    \end{enumerate}
    There are other algorithms which are easier as well as improvements to this
    algorithm. You should not solve this with an array (or string). That
    defeats the purpose of the assignment.
\end{itemize}
%
%
I provided you with a test file to test your code. You should ensure that your
code satisfies the tester's requirements. It is the tool I will use to grade
your submissions. I will only change the input and expected values. \\
%
To utilize the tester, you will need access to a python3 interpreter. The tester
can be called using the make file, assuming that \texttt{python3} is in your
path and that your present working directory is \texttt{../hw1}
%
\section*{Point Awards}
Correctly converting and printing a valid value is worth 2 points. Correctly
catching an incorrect number of bits is worth $\frac{1}{2}$ points. Correctly
catching negative decimal input is worth $\frac{1}{2}$ points. Style is worth 2 points.
\vspace{1.0em}
%
I have provided you a make file. You should definitely read the
makefile and I would encourage you to read the python tester.\\
\\
%
Late assignments will lose 25\% per day late, with no assignment begin accepted
after 4 days (100\% reduction in points).\\
\\
%
Check your syllabus for further caveats of grading.
\end{document}

